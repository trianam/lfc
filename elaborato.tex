\documentclass[a4paper,twosides]{report}
\usepackage[utf8]{inputenc}
\usepackage{lmodern}
\usepackage[T1]{fontenc}
\usepackage[italian]{babel}
\usepackage{microtype}
\usepackage{acronym}
\usepackage{mathtools}
\usepackage{amsfonts}
\usepackage[hidelinks,breaklinks=true]{hyperref}
\usepackage{xcolor}
\usepackage{listings}

\newcommand{\me}{\ensuremath{\mathrm{e}}}
\newcommand{\md}{\ensuremath{\mathrm{d}}}
\newcommand{\expected}[1]{\ensuremath{\mathrm{\textbf{E}}\left[#1\right]}}
\newcommand{\variance}[1]{\ensuremath{\mathrm{\textbf{Var}}\left(#1\right)}}
\newcommand{\prob}[1]{\ensuremath{\mathrm{\textbf{P}}\left(#1\right)}}
\newcommand{\abs}[1]{\ensuremath{\left|#1\right|}}
\newcommand{\codei}[1]{\texttt{#1}}

\lstdefinestyle{customPy}{
 language=python,
 showstringspaces=false,
 basicstyle=\footnotesize\ttfamily,
 keywordstyle=\bfseries\color{green!40!black},
 commentstyle=\itshape\color{purple!40!black},
 identifierstyle=\color{blue},
 stringstyle=\color{orange},
}

%\DeclarePairedDelimiter\abs{\lvert}{\rvert}

\author{
  {\Large Stefano Martina}\\
  {\small stefano.martina@stud.unifi.it}\\
  Universit\`a degli Studi di Firenze\\
  Scuola di Scienze Matematiche, Fisiche e Naturali\\
  Corso magistrale di Informatica
}
\title{{\Huge\bfseries Cenni di fisica statistica}\\{\large\bfseries
    Esame di Laboratorio di Fisica Computazionale}}

\begin{document}
\maketitle
\thispagestyle{empty}
\vfill
\begin{abstract}
  Questo lavoro presenta una breve introduzione alla fisica
  statistica.
\end{abstract}
\clearpage
\acresetall

\graphicspath{{img/}}

\chapter{\acf{MCM}}
\section{Note sulla probabilit\`a}
\subsection{Errore probabile}
Per una variabile casuale $X$ distribuita normalmente con
\begin{itemize}
\item media $\mu$;
\item varianza $\sigma$;
\end{itemize}
si ha che per
\begin{equation*}
  r = 0.6745\sigma
\end{equation*}
vale
\begin{eqnarray*}
  \prob{\mu-r<X<\mu+r} &=& 0.5\\
  \prob{\abs{X-\mu}<r} &=& 0.5\\
  \prob{\abs{X-\mu}>r} &=& 0.5.
\end{eqnarray*}
Quindi valori di $X$ che deviano da $\mu$ pi\`u o meno di $r$ hanno
stesse probabilit\`a ed $r$ identifica l'errore pi\`u probabile di
una distribuzione normale.

\subsection{Teorema del limite centrale delle probabilit\`a}
Si considerino $N$ variabili casuali $X_1,X_2,\dots,X_N$ indipendenti e distribuite
identicamente. Quindi con stessa media e stessa varianza:
\begin{eqnarray*}
  \expected{X_1}=\expected{X_2}=\dots=\expected{X_N}&=&m\\
  \variance{X_1}=\variance{X_2}=\dots=\variance{X_N}&=&b^2.
\end{eqnarray*}

Si consideri la somma di queste variabili casuali:
\begin{equation*}
  Y = X_1+X_2+\cdots+X_N
\end{equation*}
si ha che
\begin{eqnarray*}
  \expected{Y}&=&\expected{X_1+X_2+\cdots+X_N}=Nm\\
  \variance{Y}&=&\variance{X_1+X_2+\cdots+X_N}=Nb^2.
\end{eqnarray*}

Si consideri adesso una variabile casuale $Z$ distribuita normalmente
con parametri:
\begin{eqnarray*}
  \mu&=&Nm\\
  \sigma&=&b\sqrt{N}
\end{eqnarray*}
con funzione di densit\`a denotata da $p_Z(x)$.

Il teorema centrale del limite afferma che per $N$ grande, e per ogni
intervallo $(x_1,x_2)$ vale:
\begin{equation}\label{eq:tcl}
  \prob{x_1<Y<x_2}\approx\int_{x_1}^{x_2}p_Z(x) \md x
\end{equation}
quindi la somma di un numero elevato di variabili
casuali aventi distribuzione identica \`e distribuita normalmente con
media e varianza uguali a quelle originali (anche se queste non erano
distribuite normalmente).

\section{\acf{MCM}}
Supponendo di dover calcolare una certa quantit\`a sconosciuta $m$,
dobbiamo trovare una variabile casuale $X$ tale che:
\begin{equation*}
  \expected{X} = m.
\end{equation*}
Avendo tale distribuzione, con varianza uguale a:
\begin{equation*}
  \variance{X} = b^2
\end{equation*}
 \`e possibile mostrare i seguenti passaggi.

Si considerino $N$ variabili casuali $X_1,X_2,\dots,X_N$ con
distribuzione identica a quella di $X$. Quindi per il teorema del
limite centrale~\eqref{eq:tcl} si ha, per $N$ sufficientemente grande, che la
distribuzione della somma
\begin{equation*}
  Y=X_1+X_2+\cdots+X_N
\end{equation*}
\`e distribuita normalmente con parametri
\begin{eqnarray*}
  \mu &=& Nm\\
  \sigma&=&b\sqrt{N}.
\end{eqnarray*}

Per la regola dei \emph{tre sigma} si ha:
\begin{eqnarray*}
  \prob{\mu-3\sigma < Y <\mu +3\sigma}&\approx& 0.997\\
  \prob{Nm-3b\sqrt{N} < Y < Nm+3b\sqrt{N}}&\approx& 0.997
\end{eqnarray*}
dividendo per $N$:
\begin{eqnarray*}
  \prob{m-\frac{3b}{\sqrt{N}} < \frac{Y}{N} <
    m+\frac{3b}{\sqrt{N}}}&\approx& 0.997\\
  \prob{\abs{\frac{Y}{N}-m} <\frac{3b}{\sqrt{N}}}&\approx& 0.997
\end{eqnarray*}
e quindi:
\begin{equation}\label{eq:mc}
  \prob{\abs{\frac{1}{N}\sum_{i=1}^NX_i-m} <\frac{3b}{\sqrt{N}}}\approx 0.997.
\end{equation}

L'equazione~\eqref{eq:mc} afferma che, se viene estratto un campione per ogni
variabile casuale $X_i$, la media aritmetica di questi valori \`e
approssimativamente uguale a $m$. Inoltre, l'errore di questa
approssimazione \`e pari a $3b/\sqrt{N}$, che tende a $0$ al crescere
di $N$. \`E possibile ridurre ulteriormente l'incertezza
aumentando il numero $k$ di sigma usati per l'approssimazione e
quindi valutando l'errore $kb/\sqrt{N}$.

In realt\`a, poich\'e le variabili casuali $X_i$ hanno la stessa
distribuzione di $X$, \`e sufficiente estrarre $N$ campioni da $X$ per
giungere alle stesse conclusioni.
Il \ac{MCM} \`e costituito
dal seguente procedimento da adattare secondo i casi:
\begin{enumerate}
\item si trova la distribuzione $X$ con media la quantit\`a desiderata
  $m$ e varianza $b^2$;
\item si estraggono $N$ campioni di $X$, in numero grande abbastanza
  da avere un errore piccolo a piacere;
\item la media aritmetica di tali $N$ campioni d\`a il valore
  approssimato di $m$.
\end{enumerate}
Quindi il problema diventa quello di ricavare la distribuzione $X$ o,
comunque, gli $N$ campioni distribuiti in accordo con $X$.

Se vogliamo caratterizzare pi\`u nel dettaglio l'errore commesso
prendendo $N$ campioni, possiamo ricorrere all'\emph{errore
  probabile}. Ponendo $k=0.6745$ si ha che 
\begin{equation*}
  \prob{\abs{\frac{1}{N}\sum_{i=1}^NX_i-m} <\frac{0.6745\cdot b}{\sqrt{N}}}\approx 0.5
\end{equation*}
e quindi 
\begin{equation*}
  r_N = \frac{0.6745\cdot b}{\sqrt{N}}
\end{equation*}
 indica di quanto si discosta mediamente il
valore della media dei campioni $\frac{1}{N}\sum_{i=1}^NX_i$ dalla misura cercata
$m$. Tale valore caratterizza l'errore assoluto
\begin{equation*}
\abs{\frac{1}{N}\sum_{i=1}^NX_i-m}  
\end{equation*}
commesso prendendo $N$ campioni.

\subsection{\acf{MCI}}
Il \ac{MCM} \`e utile per simulare fenomeni che hanno un alto grado di
incertezza negli input o un alto numero di gradi di libert\`a.

Una
applicazione specifica \`e \ac{MCI}. Integrare numericamente una
funzione che ha
molte dimensioni o che ha caratteristiche particolari, come quella di
figura~\ref{fig:func}, pu\`o diventare un problema estremamente
difficile. Monte Carlo pu\`o essere utile in questi casi per ottenere
un'approssimazione dell'integrale.
\begin{figure}[h!]
  \centering
  \includegraphics[width=\textwidth]{montecarlo1.pdf}
  \caption{Funzione $y=sen^2(1/x)$ non ben definita intorno allo $0$.}
  \label{fig:func}
\end{figure}

Consideriamo la funzione:
\begin{equation*}
  f(x)\equiv sen^2(\frac{1}{x}),
\end{equation*}
\`e sempre compresa tra $0$ e $1$, non \`e definita per
$x=0$ e inoltre intorno allo $0$ si addensano le oscillazioni. La
funzione integrale
\begin{equation*}
  I(x) \equiv \int_0^x f(x') \md x'
\end{equation*}
\`e l'area sotto la curva $f(x)$ compresa tra $0$ e $x$. Ha sempre valore finito
compreso tra $0<I(x)<X$, ma non \`e semplice da calcolare vicino
all'origine.

Se scegliamo due numeri casuali
\begin{itemize}
\item  $u$ uniformemente distribuito tra $0$ e $x$;
\item $v$ uniformemente distribuito tra $0$ e $1$;
\end{itemize}
si ha che la probabilit\`a che il punto $h = (u, v)$, nella
figura~\ref{fig:func}, sia sotto la curva 
$f(x)$ \`e $I(x)/x$, ossia il rapporto tra l'area sotto la curva e
l'area limitata da $0$ e $x$ nelle
ascisse e da $0$ e $1$ nelle ordinate.

Il punto $h$ \`e sotto la
curva se $v<f(u)$. Quindi, prendendo un campione sufficientemente
grande di $N$ punti e contando gli $M$ punti che stanno sotto la
curva\footnote{Usando la disequazione $v<f(u)$.}, \`e possibile
stimare il valore di $I(x)$ con:
\begin{equation*}
  I(x)\approx \frac{M}{N}x.
\end{equation*}

In figura~\ref{fig:int} \`e possibile vedere l'approssimazione di
$I(x)$ usando \ac{MCI} con $10000$ campioni. In appendice \`e
riportato il codice\footnote{Notare che le figure
  sono state create impostando \codei{numCampioni} e \codei{numPuntiX}
  uguali a 10000, mentre nel codice sono uguali a 1000 per velocizzare
  il processo in caso di esecuzione di questo.} \emph{Python} usato
per creare i plot delle
figure~\ref{fig:func} e~\ref{fig:int}.
\begin{figure}[h!]
  \centering
  \includegraphics[width=\textwidth]{montecarlo2.pdf}
  \caption{Approssimazione della funzione integrale $y=\int_0^x sen^2(1/x')\md x'$, ottenuta
    con \ac{MCI} con $N=10000$ campioni.}
  \label{fig:int}
\end{figure}

\newpage
\section{Appendice}
\lstinputlisting[style=customPy]{montecarlo.py}


\chapter{\acf{SA}}
\section{Introduzione}
Il \ac{SA} \`e un metodo usato per trovare l'ottimo globale di una
funzione data, si ispira ad un metodo usato in metallurgia chiamato
annealing che consiste nello scaldare e raffreddare lentamente un
materiale per aumentarne la dimensione dei cristalli e migliorarne le
caratteristiche chimico-fisiche.

La funzione da ottimizzare pu\`o avere diverse variabili, un tipico
esempio di problema per il quale pu\`o essere usato \ac{SA} \`e
\ac{TSP}.

\section{Termodinamica statistica}
Per descrivere i principi base della termodinamica statistica si
considera un sistema di esempio. In un reticolo monodimensionale ogni
punto \`e una particella con un valore di spin che pu\`o essere
\emph{up} o
\emph{down}. Se il reticolo ha $N$ punti, allora il sistema pu\`o
essere in $2^N$ diverse configurazioni, ad ognuna di queste
configurazioni corrisponde un valore di energia, ad esempio:
\begin{equation*}
  E=B(n_+-n_-)
\end{equation*}
dove $B$ \`e una certa costante, $n_+$ \`e il numero di particelle con
spin 
\emph{up}, ed $n_-$ \`e il numero di particelle con spin \emph{down}.

La probabilit\`a $P(\sigma)$ di trovare il sistema in una certa
configurazione
$\sigma$ \`e data dalla distribuzione di Boltzmann-Gibbs:
\begin{equation}\label{eq:distBoltz}
  P(\sigma) = C \me^{-E_\sigma/T}
\end{equation}
dove $E_\sigma$ \`e l'energia della configurazione, $T$ \`e la
temperatura\footnote{Nella distribuzione di Boltzmann-Gibbs al
  denominatore dell'esponente vi \`e $kT$ dove $k$ \`e la costante di
  Boltzmann e $T$ \`e la temperatura termodinamica. Ma per l'esempio
  la temperatura \`e un parametro svincolato dalla realt\`a fisica
  quindi \`e possibile trascurare $k$.} e $C$ \`e una costante di
normalizzazione.

L'energia media del sistema in equilibrio \`e quindi:
\begin{eqnarray*}
  \bar{E} &=& \frac{\sum_\sigma E_\sigma P(\sigma)}{\sum_\sigma
    P(\sigma)}\\
  &=& \frac{\sum_\sigma E_\sigma \me^{-E_\sigma/T}}{\sum_\sigma \me^{-E_\sigma/T}}
\end{eqnarray*}
Calcolare numericamente il valore di $\bar{E}$ pu\`o essere
difficile al crescere del numero degli stati, per\`o \`e possibile
creare un algoritmo di Monte Carlo che simula le fluttuazioni casuali
tra gli stati in modo che la distribuzione data
dall'equazione~\eqref{eq:distBoltz} sia rispettata. Partendo da una
configurazione iniziale arbitraria, dopo un certo numero di \emph{trial} di Monte
Carlo il metodo converge verso lo stato di
equilibrio $\bar{E}$ e continua ad oscillare intorno ad $\bar{E}$.

\ac{SA} \`e un metodo di questo tipo.

\section{Algoritmo \acf{SA}}
Il \ac{SA} si muove su un sistema partendo da un certo stato iniziale
$s_0$, quindi esegue una serie di iterazioni nelle quali viene
valutato un vicino dello stato e, con una certa distribuzione di
probabilit\`a, la dinamica si sposta nel nuovo stato vicino o meno.

Un possibile codice \emph{c-style} di un metodo \ac{SA} \`e il seguente:
\begin{lstlisting}[frame=single, numbers=left, language=C]
s = s0;
for(int k=0; k<kMax; ++k) {
  T = temp(k/kMax);
  sNew = vicino(s);
  if (random(0,1) < P(E(s), E(sNew), T)) {
    s = sNew;
  }
}
return s;
\end{lstlisting}
In riga \codei{1} viene inizializzato il ciclo con lo stato iniziale
\codei{s0}, in seguito vi \`e un ciclo che esegue \codei{kMax}
iterazioni. In riga \codei{3} viene assegnato un certo
valore di tempo a \codei{T} che dipende dal progresso delle
iterazioni. In riga \codei{4} viene scelto casualmente un vicino \codei{sNew} dello
stato \codei{s}. In riga \codei{5} vi \`e il fulcro dell'algoritmo,
\codei{random} sceglie uniformemente un numero casuale in $[0,1)$ e
\codei{P} implementa la distribuzione di probabilit\`a di accettazione
che dipende dall'energia
misurata in \codei{s}, da quella in \codei{sNew} e dalla temperatura
\codei{T}. In caso di accettazione viene preso \codei{sNew} come nuovo
stato di base e continua il processo.

La relazione con la termodinamica statistica sta nel fatto che
\codei{P} viene scelta in modo da mantenere vera la
relazione~\eqref{eq:distBoltz}\footnote{\`E sufficiente che siano relazioni simili.},
inoltre il metodo \codei{temp} ritorna
valori decrescenti di temperatura al crescere delle iterazioni, da qui
il paragone con l'annealing in metallurgia.

Originalmente, per
rispettare i principi della termodinamica statistica, \codei{P} venne
scelta in modo da ritornare \codei{1} se
$E(sNew)<E(s)$ e $\me^{-(E(sNew)-E(s))/T}$ altrimenti. Tale
condizione non \`e strettamente necessaria per realizzare dei metodi
\ac{SA}.

\section{\acf{TSP}}
\ac{TSP} \`e un problema di ottimizzazione combinatoria e verr\`a
usato come esempio per descrivere il metodo \ac{SA}. Vengono dati:
\begin{itemize}
\item una lista di $N$ citt\`a identificate dai naturali da $1$ ad $N$;
\item una matrice $D$, $N\times N$, dove $D_{i,j}$ \`e la distanza, o
  costo, per andare dalla citt\`a $i$ alla citt\`a $j$;
\item si indica con $\{x_i\}_1^N$ una permutazione $x$ dei naturali da
  $1$ a $N$, ovvero delle citt\`a.
\end{itemize}
L'obbiettivo del problema \`e di trovare una permutazione $c$ delle
citt\`a tale che sia minima la distanza $d$ espressa dall'equazione:
\begin{equation}\label{eq:distanzaTSP}
  d(c) = D_{c_N,c_1}+\sum_{k=1}^{N-1}D_{c_k,c_{k+1}}
\end{equation}

Ossia si chiede di trovare un percorso che:
\begin{enumerate}
\item passi da tutte le citt\`a
  una ed una sola volta;
\item sia ciclico, ossia finisca nella citt\`a da cui \`e cominciato;
\item sia quello con lunghezza minima tra quelli che rispettano le due
  precedenti condizioni.
\end{enumerate}

\section{Soluzione \acf{TSP} con \acf{SA}}
Per risolvere il problema \ac{TSP} vengono interpretati come stati le
diverse permutazioni delle $N$ citt\`a e viene interpretata la distanza data
dalla formula~\eqref{eq:distanzaTSP} come energia del sistema quando
si trova in un certo stato, ossia permutazione delle citt\`a.
Lo pseudo codice che implementa il metodo \`e il seguente:
\begin{lstlisting}[frame=single, numbers=left]
tsp(s, T) {
  c[k] = s[k] per k=1,...,N
  dc = distanza(c)
  i = 1
  while (! stop()) {
    j = random(1,N) diverso da i
    t = scambia(c, i, j)
    dt = distanza(t)
    if ((dt<dc) or (random(0,1)<exp((dc-dt)/T))) {
      c[k] = t[k] per k=1,...,N
      dc = dt
    }
    i = modulo(i, N) + 1
  }
  return c
}
\end{lstlisting}
dove il metodo \codei{distanza(s)} calcola la distanza di un percorso
rappresentato da una permutazione \codei{s}:
\begin{lstlisting}[frame=single, numbers=left]
distanza(s) {
  return D[s[N],s[1]] + sum(D[s[k], s[k+1]])
                             per k=1,...,N-1
}
\end{lstlisting}
\codei{scambia(c, i, j)} scambia \codei{i} con \codei{j} nel
percorso \codei{c} e inverte la direzione di quelli in mezzo a
\codei{i} e \codei{j}:
\begin{lstlisting}[frame=single, numbers=left]
scambia(c, i, j) {
  a = min(i,j)
  b = max(i,j)
  t[k] = c[k]      per k=1,...,a-1
  t[a+k] = c[b-k]  per k=0,...,b-a
  t[k] = c[k]      per k=b+1,...,N
  return t
}  
\end{lstlisting}
\codei{i = modulo(i,N)+1} serve per iterare \codei{i} in $1,\dots,N$
ripetutamente.

Nel codice visto manca da definire la condizione di stop
data dal metodo \codei{stop()}, essa dovrebbe avvenire quando viene
raggiunto un equilibrio, ossia quando il metodo oscilla intorno ad un
certo stato. L'idea \`e che inizialmente si lancia
la procedura con uno stato iniziale \codei{s} casuale e con una
temperatura \codei{T} casuale o scelta con un certo criterio. L'algoritmo si ferma
quando viene raggiunto un equilibrio e ritorna lo stato \codei{c} in
quell'equilibrio. A questo punto viene decrementato opportunamente la
temperatura
\codei{T} e viene eseguita nuovamente la procedura chiamandola sui due nuovi valori
\codei{TSP(c,T)}.


\chapter*{Acronimi}
\addcontentsline{toc}{section}{Acronimi}
\begin{acronym}[MCM]
  \acro{MCM}{Monte Carlo Method}
  \acro{MCI}{Monte Carlo Integration}
  \acro{SA}{Simulated Annealing}
  \acro{TSP}{Traveling Salesman Problem}
\end{acronym}

\nocite{*}
\phantomsection
\addcontentsline{toc}{section}{\refname}
\bibliographystyle{apalike}
\bibliography{elaborato}

\end{document} 
