\chapter{\acf{LMM} e \acf{LRM}}
In questo capitolo vengono presi in considerazione due metodi che
permettono di calcolare in modo semplice la soluzione a problemi
di massimizzazione o minimizzazione di funzioni multivariate
sottoposte a vincoli. Ad esempio problemi come quello di trovare il
punto più elevato di una strada di campagna (massimizzare la funzione
bivariata che rappresenta l'elevazione del terreno sottoposta al
vincolo rappresentato dalla strada), oppure trovare il punto di minima
pressione all'interno del perimetro regionale (minimizzare la funzione
bivariata che rappresenta la carta delle isobare sottoposta al vincolo
dato dai confini della regione).
\section{\acf{LMM}}
Il \ac{LMM} \`e una strategia che permette di trovare il massimo o il
minimo di una funzione sottoposta a vincoli dati da
equazioni. \`E utile analizzare prima il caso con un solo vincolo.
\subsection{Singolo vincolo}
Viene indicato con $f$ la funzione da
massimizzare/minimizzare e con $g$ il vincolo, per semplificare il
problema viene mostrato il caso in due variabili in cui:
\begin{eqnarray*}
f&:&\mR^2\rightarrow\mR\\
g&:&\mR^2\rightarrow\mR.
\end{eqnarray*}
Senza perdere generalit\`a viene preso in considerazione il problema
in cui \`e necessario massimizzare $f$, quindi formalmente il problema
\`e:
\begin{defi}
  \begin{eqnarray*}
    &&\max_{x,y}\left\{f(x,y)\right\}\\
    &&\tc{}\\
    &&\quad g(x,y)=0.
  \end{eqnarray*}
\end{defi}
Per applicare il metodo \`e necessario che $f$ e $g$ siano derivabili
alle prime derivate parziali in $x$ e $y$.

L'intuizione del metodo \`e quella di trovare i punti critici in cui
non \`e possibile aumentare $f$ nell'intorno della $g$.
\section{\acf{LRM}}